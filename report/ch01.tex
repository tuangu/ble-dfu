
\section{Introduction}
\justify
This chapter introduces the motivation of the thesis work. The first two sections 
presents the current interests in embdded devices and the need for device firmware
update. Afterward, the last sections states the purposes and goals of the project.

\subsection{Increasing interest in embedded devices}
\justify
Embedded devices are highly specialised devices meant for one or a 
few special purposes. They are usually embedded into a larger object 
or part of bigger systems that serves greater purposes. The hardware 
of these devices is relatively simple and small featuring, for example, 
a 32-bit microcontroller, a dedicated processor called an application-specific 
integrated circuit, or digital signal processor. The software is also 
differing from application software. Unlike application software, embedded 
software is developed toward a specific hardware in which the capabilities 
and the addition of third-party components are strictly controlled. It may 
use either no operating system or a real-time operating system. Embedded 
devices have extensive applications in commercial, consumer, industrial, 
automotive, health-care and many other industries because of their diminutive 
and inconspicuous nature. In the late of the twentieth century and the 
beginning of the twenty-first century, the development of the Internet 
and the advanced in material science brings the potential to computerise 
and connect industrial and consumer components such as car, manufacturing 
plant, wristwatch, or even home kitchen. 

\justify
For example, the Internet of Things (IoT) is an emerging topic of technical, 
social, and economic significance. The term IoT general refers to the 
scenarios where objects, sensors, and everyday items with networking 
connectivity gather data about their surrounding environment so that they 
can be used to analyse, interpret, decide, and act on that data and other 
associated information \cite{KSL15}. IoT allows devices to generate, exchange, 
and consume data to form a fully interconnected “smart” world operating with 
minimal human intervention toward predefined goals. As a result, physical objects 
and machines are being gradually computerised by, for example, adding 
networking connectivity, to take advantage of the transformational 
potential of the IoT. 

\subsection{The need for firmware update}
\justify
As the number of devices connect to the internet increase, new opportunities 
to exploit the vulnerabilities also grow. IoT devices, in particular, are 
an ideal candidate for attackers, for several reasons: First, the device 
is connected to the internet, so attackers do not have to have physically 
access to exploit. Second, many IoT devices are operated unattended which 
means that the attacks might not be recognised until something significant 
happens. Third, embedded systems are inherently difficult to update, which 
leaves them more vulnerable to newly-discover security issues.  Hence, 
new IoT devices should be developed with upgradability in mind that allows 
the delivery manufacturers regular software updates from manufactures that 
not only patch bugs and enhance security but also deliver new features 
and functionalities.

\subsection{Purposes and goals}
\justify
The first part of the project is to study various device firmware update 
mechanisms that have been implemented and deployed. Then a proof-of-concept 
application will be developed that features device firmware update over 
Bluetooth Low Energy for an IoT device to demonstrate the feasibility and 
practicality in real-world scenarios.  